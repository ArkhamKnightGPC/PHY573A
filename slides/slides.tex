\documentclass{beamer}
\usetheme[progressbar=frametitle]{metropolis}
\usepackage{appendixnumberbeamer}

\usepackage{booktabs}
\usepackage[scale=2]{ccicons}

\usepackage{pgfplots}
\usepgfplotslibrary{dateplot}

\usepackage{xspace}
\newcommand{\themename}{\textbf{\textsc{metropolis}}\xspace}

\usefonttheme{structurebold}
\usecolortheme{orchid}

\usepackage{preamble}

\title{Micro and Nano Electronics Experimental Design}
\author{\textbf{Léo Gabriel \& Gabriel Pereira de Carvalho}}
\institute{PHY\_51173\_EP \\
Professor Costel Sorin Cojocaru \\ École polytechnique}
\date{\today}

\AtBeginSection[]{
  \begin{frame}
  \vfill
  \centering
  \begin{beamercolorbox}[sep=8pt,center,shadow=true,rounded=true]{title}
    \usebeamerfont{title}\insertsectionhead\par%
  \end{beamercolorbox}
  \vfill
  \end{frame}
}

\begin{document}
\frame{\titlepage}

\section{Introduction}

\begin{frame}{The end of Moore's law ? State of CMOS technology}
\begin{itemize}
    \item CMOS technology approaching physical and practical limits:
    \begin{itemize}
        \item higher power densities (end of Dennard scaling in the 2000s);
        \item rising fabrication costs;
        \item increased leak currents;
    \end{itemize}
\end{itemize}

\begin{figure}
\subfigure[IBM's 2nm chip (2021)]{
    \includegraphics[width=0.35\linewidth]{slides/IBM Research 2 nm Wafer_678x452.jpg}
    \label{fig:IBM}
}
\subfigure[Moore's law: number of transistors in an IC doubles every 2 years]{
    \includegraphics[width=0.35\linewidth]{slides/moore.png}
    \label{fig:moore}
}
\end{figure}
\end{frame}

\begin{frame}{Consequences of the end of Moore's law}
“The only way to get more computing capacity today is to build bigger, more energy-consuming machines. If we’re in an AI arms race with our adversaries, it could have a dramatically bad impact on climate.” \textbf{Professor Charles Leiserson, MIT}
%programmers have grown accustomed to consistent improvement in performance being a given, which has led to practices that valued productivity over performance. 
\begin{figure}
    \centering
    \includegraphics[width=0.5\linewidth]{slides/Leiserson.jpg}
    \label{fig:MIT}
\end{figure}
\end{frame}


%Unlike silicon-based transistors, which are made at temperatures around 450 to 500 degrees Celsius, CNFETs also can be manufactured at near-room temperatures. “This means that you can actually build layers of circuits right on top of previously fabricated layers of circuits, to create a three-dimensional chip,” Shulaker explains. “You can’t do this with silicon-based technology, because you would melt the layers underneath.”

%A 3D computer chip, which might combine logic and memory functions, is projected to “beat the performance of a state-of-the-art 2D chip made from silicon by orders of magnitude,” he says.

%\begin{itemize}
%    \item electron mobility close to theoretical limit
%    \item nanoscale dimensions
%    \item low threshold voltage $\imply$ faster switching
%    \item allow fabrication of 3D integrated circuits
%\end{itemize}
\begin{frame}{Can CNFETs restore Moore's law ?}
\begin{itemize}
    \item Carbon nanotube replaces silicon channel
    \item Ballistic electron transport (electron mobility approaches theoretical limit)
    \item Faster switching speed %because of nanoscale dimensions and low Vth
    \item Lower parasitic losses %fixes high dynamic consumption of CMOS
\end{itemize}
\begin{figure}[h]
    \centering
    \subfigure[CNTFET principle]{\label{fig:CNTFET-principle}\includegraphics[width=0.40\linewidth]{0_Principle/CNTFET_Principle.png}}
    \subfigure[A substrate for CNTFET with 6 illustrative devices]{\label{fig:CNTFET-principle2}\includegraphics[width=0.40\linewidth]{0_Principle/CNTFET_Principle2.png}}
\end{figure}
\end{frame}

%\begin{frame}{Carbon nanotubes}
%    \begin{itemize}
%        \item A type of \textit{fullerene}
%        \item Carbon "tubes" with graphene structure
%        \item Two main characteristics: diameter, chirality
%        \item Can be conductor or semiconductor
%    \end{itemize}
%    \begin{figure}
%        \centering
%        \includegraphics[width=0.5\linewidth]{slides/Carbon_nanotube_zigzag_povray_cropped.PNG}
%        \caption{Carbon nanotube representation}
%        \label{fig:enter-label}
%    \end{figure}
%\end{frame}

\section{Metal catalyst deposition}

\begin{frame}{Electron-beam vapor deposition}
\begin{itemize}
    \item Heated filament emits electrons
    \item Electrons accelerated by an electric field (flux) heat up metal
    \item Sublimated metal: homogeneous deposit on wafer
    \item Start / stop by moving the shutter
    \item Vacuum ($10^{-8} mbar$): Sublimation and no electron absorbed by air
\end{itemize}
\begin{figure}[h]
    \centering
    \label{fig:Evap}\includegraphics[width=0.3\linewidth]{1_Deposit/deposit_principle.png}
    %\caption{Electron-beam vapor deposition diagram}
\end{figure}
\end{frame}

\begin{frame}{Thickness of metallic catalyst layers}
\begin{itemize}
    \item Full monolayer of alumina and $0.1$ monolayer equivalent of iron \\ $\implies$ iron atoms cover about $20\%$ of wafer surface (good mobility)
\end{itemize}
\begin{align}
    \begin{cases}
        \epsilon_{\mathrm{Al_2O_3}} =  1 \mathrm{nm}\\
        \epsilon_{\mathrm{Fe}} = 0.15 \mathrm{nm}
    \end{cases}
\end{align}
\begin{figure}
    \centering
    \includegraphics[width=0.5\linewidth]{photos_lab/metal_deposition.jpg}
    %TALK ABOUT VARIATIONS IN THESE PARAMETERS!!!!
    \label{fig:waf}
\end{figure}
\end{frame}

\begin{frame}{Computing deposition time}
\[
    \epsilon = I_{\mathrm{flux}} \cdot t \cdot C_{\text{evap}}
\]
where $C_{\text{evap}}$ is a constant value which is given for each metal.

\begin{table}[H]
\centering
\scriptsize
\begin{tabular}{|c|c|c|}
    \hline
    \textbf{Materials} & Al$_2$O$_3$ & Fe \\
    \hline
    \textbf{Crucible position (x,y,z)}  & (19.5, -6, -2) & (39, 5, -2) \\
    \hline
    \textbf{Beam evaporation constant $C_{\mathrm{evap}}$ (nm$\cdot \mu$A$^{-1} \cdot$s$^{-1}$)}  & 0.046 & 0.026 \\
    \hline
\end{tabular}
\end{table}
\textbf{Example}
$$t_{Al_2O_3} = \frac{1 \mathrm{nm}}{100 \mathrm{nA} \cdot 0.046 \mathrm{nm} \cdot \mu\mathrm{A}^{-1} \cdot \mathrm{s}^{-1}} = 217 \mathrm{s} = 3 \mathrm{min} 37 \mathrm{s}$$
\end{frame}

\section{Carbon nanotube growth (CVD)}

\begin{frame}{CVD: specific properties of silica layer on the wafer}
\begin{itemize}
    \item Very smooth surface $\implies$ uniform deposition of the metal particles, regular clusters
    \item Stable in the temperature range inside vacuum chamber (up to 800$^{\circ}$)
    \item Chemically inert
\end{itemize}
\end{frame}

\begin{frame}{How to form carbon nanotubes ?}
\begin{itemize}
    \item Graphene and nanotubes have $sp^2$ hybridization
    \item Graphene is flat because the $120^{\circ}$ angles in the hexagons match the angle separating the $sp$ orbitals
    \item We use metal catalysts to introduce defects in carbon structure during deposition!
\end{itemize}
\begin{figure}[h]
    \centering
    \subfigure[CNT structure]{\label{fig:CNT}\includegraphics[width=0.28\linewidth]{photos_lab/sp2.PNG}}
    \subfigure[Defects on graphene structure]{\label{fig:fullerene}\includegraphics[width=0.50\linewidth]{slides/fullerene.jpg}}
\end{figure}
\end{frame}

\begin{frame}{CVD procedure}
\begin{itemize}
    \item $10^{-6} mbar$ vaccum (avoid pollution), then introduce gases ($3 mbar$)
    \item Holes + heating up 
    \begin{itemize}
        \item H$_2$ mass flow controller $\implies$ flow of 30 SCCM
    \end{itemize}
    \item Growth
    \begin{itemize}
        \item H$_2$ mass flow controller $\implies$ flow of 20 SCCM
        \item CH$_4$ mass flow controller $\implies$ flow of 10 SCCM
    \end{itemize}
    \item Methane deposition duration is critical
    \item Maintain $T= 800\text{°}C$, $p = 3 mbar$ critical
    \item Stop: pressure kills the reaction
\end{itemize}

\begin{figure}
    \centering
    \rotatebox{0}{\includegraphics[width=0.2\linewidth]{photos_lab/CVD.jpg}}
    %\caption{Vacuum chamber during CVD procedure}
    \label{fig:CVD}
\end{figure}
\end{frame}

\begin{frame}{Verify CNT growth: Raman Spectroscopy}
    \begin{itemize}
        \item Silica line
        \item G-band \(\rightarrow\) graphene structure
        \item D-band \(\rightarrow\) $sp^2$ (clue for CNT)
    \end{itemize}
    \begin{figure}[H]
        \centering
        \subfigure[\scriptsize{1 nm Al$_2$O$_3$, 0.1 nm Fe (30min CVD)}]{\label{fig:01Fe30min}\includegraphics[width=0.45\linewidth]{Raman/RAMAN_iron01_30min.png}}
        \subfigure[\scriptsize{1 nm Al$_2$O$_3$, 0.1 nm Fe (1h CVD)}]{\label{fig:01Fe1hr}\includegraphics[width=0.45\linewidth]{Raman/RAMAN_iron01_1hr.png}}
    \end{figure}
\end{frame}

\begin{frame}{Scanning electron microscopy (SEM)}
\begin{itemize}
    \item Sample with 1 nm Al$_2$O$_3$ and 0.15 nm Fe with 30 minutes CVD on the SEM
    \begin{itemize}
        \item we tried smaller Fe deposition to reduce density (0.075, 0.05 nm)
        \item we tried less active catalysts (Co and Ni)
    \end{itemize}
\end{itemize}
\begin{figure}
    \centering
    \includegraphics[width=0.5\linewidth]{photos_lab/SEM.jpg}
    \label{fig:SEM}
\end{figure}
\end{frame}

\section{Electrode deposition}

\begin{frame}{Photolitography}
    \begin{itemize}
        \item Process to deposit a material following a mask
        \item Was done at Thales research center(Plateau de Saclay)
        \item Procedure with specific timing for our application
    \end{itemize}
    \begin{figure}
        \centering
        \includegraphics[width=0.8\linewidth]{slides/photolitho_steps.png}
    \end{figure}
\end{frame}

\begin{frame}{Photolitography}
    \begin{figure}[h]
        \centering
        \subfigure[Electrodes seen with SEM]{\label{fig:CNT}\includegraphics[width=0.45\linewidth]{slides/4(1).jpg}}
        \subfigure[Sample ready]{\label{fig:fullerene}\includegraphics[width=0.45\linewidth]{slides/sample_with_electrodes(1).jpg}}
    \end{figure}
\end{frame}

\section{Electronic characterization}

\begin{frame}{Electronic characterization}
    \begin{figure}
        \centering
        \includegraphics[width=0.9\linewidth]{slides/elec_charact.png}
        \\
        A device being tested.

        \begin{itemize}
        \item Fixed $V_{GD} = 1.0V$
            \item Plot $I_{DS} = f(V_{GS})$
            \item $V_{GS}$ : sweep from $-10V$ to $+10V$ (back and forth)
        \end{itemize}
    \end{figure}
\end{frame}

\begin{frame}{Electronic characterization}
    \begin{figure}
        \centering
        \includegraphics[width=0.75\linewidth]{slides/failed_charact.png}
        \caption{Failed characterization}
        \label{fig:enter-label}
    \end{figure}
\end{frame}

\begin{frame}{Electronic characterization}
    \begin{figure}[h]
        \centering
        \subfigure[1 nm $Al_2O_3$ 0,15 nm $Fe$]{\label{fig:CNT}\includegraphics[width=0.38\linewidth]{slides/curve 1 0,15.png}}
        \subfigure[1 nm $Al_2O_3$ 0,10 nm $Fe$]{\label{fig:fullerene}\includegraphics[width=0.38\linewidth]{slides/curve 1 0,1 .png}}
        \subfigure[1 nm $Al_2O_3$ 0,05 nm $Fe$]{\label{fig:fullerene}\includegraphics[width=0.38\linewidth]{slides/curve 1 0,05.png}}
    \end{figure}
\end{frame}
\begin{frame}{Conclusion}
    \begin{itemize}
        \item We were able to make transistors
        \item Insights on the challenges of CNTFET technology
        \begin{itemize}
            \item Scalability (how to make billions of such devices)
            \item Size reduction
            \item Process complexity and time
        \end{itemize}
    \end{itemize}
    
\end{frame}

\begin{frame}{Making CNTFET}
    \centering Thank you for listening !
    
    \begin{figure}[h]
        \centering{\label{fig:CNT}\includegraphics[width=0.38\linewidth]{slides/sample_with_electrodes(1).jpg}}
    \end{figure}
\end{frame}


\end{document}